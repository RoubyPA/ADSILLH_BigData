\documentclass[12pt]{report}
%% Language and font encodings
\usepackage[francais]{babel}
\usepackage[utf8]{inputenc}
\usepackage[T1]{fontenc}
\usepackage{lmodern} 

%% Sets page size and margins
\usepackage[a4paper,top=3cm,bottom=2cm,left=3cm,right=3cm,marginparwidth=1.75cm]{geometry}

%% Useful packages
\usepackage{amsmath}
\usepackage{graphicx}
\usepackage[colorinlistoftodos]{todonotes}
\usepackage[colorlinks=true, allcolors=blue]{hyperref}

\title{Projet Licence ADSILLH 2017/2018\\Rapport}
\author{Pierre Antoine Rouby - Lunix}

\date{Année 2017/2018}

\begin{document}
\maketitle

\begin{abstract}
Ce document présente notre projet bigdata.
\end{abstract}

\tableofcontents

\chapter{Le Projet}
\section{Les besoins}
Aujourd'hui pour acheter un bien immobilier il y a deux solutions principal:
\begin{itemize}
\item Passer par du particulier à particulier (leboncoin),
\item Passer par une agence immobiliaire.
\end{itemize}
La première solution permet d'éviter les frais d'agence, mais il faut avoir
confience en la personne qui vend le bien.
En effet le cadre juridique et moins plus permisif dans le cas ou il n'y à pas
d'agence.
Il faut donc faire 2 fois plus attention au arnaque et vise cacher.

\section{Solution}
Nous avons donc réfléchie à une solution pour reduire les coups d'agence avec
de la dématerialisation des visites et de l'automatisation de l'estimation des
biens.

Il serrai en effet possible de déterminé automatiquement la valeur d'un bien
en fontion de sa géolocalisation, de ses plans et de plusieur autre facteur.

La géolocalisation permettrai de connaitre les commerces, les trasports en commun
et tous autre services à proximité, mais aussi via le registre de cadastre
\footnote{Registre des cadastre français : \url{https://cadastre.gouv.fr/scpc/accueil.do}}
de connaitre la valeur des biens vendu résament dans les autours.

Les plans quand à eux serve à connaitre la surface du bien, mais pour un véritable
estimation il faut aussi prendre en compte l'état actuelle des mures, sols,
plafont, port, etc. Pour sela nous pouvons imaginé utilisé une camera à 360 à
fin de modélisé l'apartement en 3D. Ses images pourons aussi servir a fair des
visites de l'apartement avec un casque VR (Virtual Reality).
%% Add images

On peut imginé utilisé du ``machin learning'' pour le calcule d'un bien, mais
il est aussi possible de simplement appliquer une formule de calcule prés
déterminer.

\chapter{Architecture}
\section{Vue d'ensemble}
\section{Données}
\subsection{Recolte}
\subsection{Stokage}
\subsection{Traitement}
\section{Interface utilisateur}

\chapter{Conclusion}
\section{Fesabilité}
\section{Impact social}

\end{document}
